\documentclass[11pt,a4paper]{article}

\usepackage{geometry}
 \geometry{
 a4paper,
 total={150mm,237mm},
 left=30mm,
 top=30mm,
 }

% cf. http://tex.stackexchange.com/questions/50182/subtitle-with-the-maketitle-page
\usepackage{titling}
\newcommand{\subtitle}[1]{%
  \posttitle{%
    \par\end{center}
    \begin{center}\large\textbf{#1}\end{center}
    \vskip0.5em}%
}

\usepackage{color}
\usepackage{graphicx}
\usepackage{subcaption}

\usepackage[utf8]{inputenc}
\usepackage[lf]{venturis} %% lf option gives lining figures as default; 
\usepackage[T1]{fontenc}
\usepackage{beramono}
\usepackage{csquotes}
\usepackage[UKenglish,german]{babel}

\usepackage{fancyvrb}

\widowpenalty10000  % http://tex.stackexchange.com/questions/4152/how-do-i-prevent-widow-orphan-lines
\clubpenalty10000

\title{The SysSon Platform}
\subtitle{Technical Report TR-2016-11-1\\Institute of Electronic Music and Acoustics, Graz\\(Status: in progress)}
\author{Hanns Holger Rutz}
% \date{09-Feb-2016}
\date{November 2016}

% cf. https://tex.stackexchange.com/questions/94126/change-font-to-only-section-and-subsection-of-my-document
%\usepackage{titlesec}
%\titleformat{\chapter}[display]
%  {\fontfamily{pag}\selectfont\huge\bfseries}
%  {\chaptertitlename\ \thechapter}
%  {20pt}
%  {\Huge}
%\titleformat{\section}
%  {\fontfamily{pag}\selectfont\bfseries\Large}
%  {\thesection}
%  {1em}
%  {}
%\titleformat{\subsection}
%  {\fontfamily{pag}\selectfont\bfseries\Large}
%  {\thesection}
%  {1em}
%  {}

\usepackage[backend=biber,authordate]{biblatex-chicago} % citereset=chapter
%\usepackage[backend=biber,natbib,isbn=false,useprefix=true,sorting=ydnt]{biblatex-chicago} % citereset=chapter
\addbibresource{all.bib} % add a bib-reference file
\addbibresource{rutz.bib} % add a bib-reference file

% warning: https://tex.stackexchange.com/questions/313477/
% \usepackage{csquotes}

\usepackage{tabularx}
% cf. https://tex.stackexchange.com/questions/84400/table-layout-with-tabularx-column-widths-502525
\newcolumntype{s}{>{\hsize=1cm}X}

% says you should load after babel and fontspec
\usepackage[shrink=10, babel=true]{microtype}	% http://tex.stackexchange.com/questions/141852/latex-allows-line-break-between-concluding-em-dash-and-comma-before-a-new-sub-cl/141854#141854

% has to come first for full scale TeX voodoo bullcrap
\usepackage{hyperref}
% get rid of the horrible coloured boxes around links
\hypersetup{
    colorlinks,%
    citecolor=black,%
    filecolor=black,%
    linkcolor=black,%
    urlcolor=black
}
% has to come after frickin hyperref
\VerbatimFootnotes

\newcommand{\todo}[1]{\colorbox{yellow}{\textsc{todo}: #1}}

\newcommand{\quot}[1]{\guillemotleft {#1}\guillemotright}

\newcommand{\worktitle}[1]{\textit{#1}}

\newcommand{\workentry}[2]{\vspace{7.5pt}\noindent\textbf{#1} (#2)}
\newcommand{\workentrySel}[2]{\vspace{7.5pt}\noindent\textbf{#1}$*$ (#2)}

\newcommand{\figref}[1]{Fig.~\ref{#1}}

\newcommand{\software}[1]{\textit{#1}}

\newcommand{\sysson}[0]{SysSon}
\newcommand{\syssonVersion}[0]{1.8.0}
\newcommand{\syssonVersionS}[0]{1.8.0-SNAPSHOT}

\newcommand{\artefacts}[0]{\textsc{Artefacts:}}
\newcommand{\assessment}[0]{\textsc{Assessment:}}

\usepackage{listings}

\definecolor{dkgreen}{rgb}{0,0.6,0}
\definecolor{gray}{rgb}{0.5,0.5,0.5}
\definecolor{mauve}{rgb}{0.58,0,0.82}

\lstdefinestyle{plain}{
  frame=tb,
  aboveskip=3mm,
  belowskip=3mm,
  showstringspaces=true,
  columns=flexible,
  basicstyle={\small\ttfamily},
  numbers=none,
  numberstyle=\tiny\color{gray},
  keywordstyle=\color{blue},
  commentstyle=\color{dkgreen},
  stringstyle=\color{mauve},
  frame=none,
  keepspaces=true,
  breaklines=true,
  breakatwhitespace=true,
  tabsize=3,
}

\lstdefinestyle{scala}{
  frame=tb,
  language=scala,
  aboveskip=3mm,
  belowskip=3mm,
  showstringspaces=true,
  columns=flexible,
  basicstyle={\small\ttfamily},
  numbers=none,
  numberstyle=\tiny\color{gray},
  keywordstyle=\color{blue},
  commentstyle=\color{dkgreen},
  stringstyle=\color{mauve},
  frame=none,
  keepspaces=true,
  breaklines=true,
  breakatwhitespace=true,
  tabsize=3,
}

\lstdefinestyle{scala-small}{
  frame=tb,
  language=scala,
  aboveskip=3mm,
  belowskip=3mm,
  showstringspaces=true,
  columns=flexible,
  basicstyle={\tiny\ttfamily},
  numbers=none,
  numberstyle=\tiny\color{gray},
  keywordstyle=\color{blue},
  commentstyle=\color{dkgreen},
  stringstyle=\color{mauve},
  frame=none,
  keepspaces=true,
  breaklines=true,
  breakatwhitespace=true,
  tabsize=3,
}

\begin{document}
% \begin{titlepage}
\maketitle
\selectlanguage{UKenglish}
\thispagestyle{empty}
\newpage
\section{QBO - Blob Sonification}

After some initial experiments with frequency modulation to indicate blob slice height, it was decided to try out other forms of timbre modification such as wave-shaping. SuperCollider provides wave-shaping by means of the \Verb!Shaper! UGen, typically with a buffer prepared with Chebychev polynomial functions. However, it seems not possible to continuously fade in a particular timbre by altering the input signal's amplitude, as different lower partials will transitorily be attenuated. Another possibility is through the \Verb!VOsc! variable table oscillator. In order to generate the appropriate wave-tables, a graph element \Verb!BufferGen! has been added to \software{Sound\,Processes}. As \Verb!VOsc! depends on a trick of allocating multiple buffers with consecutive identifiers, and this consecutiveness is currently not possible to guarantee in \software{Sound\,Processes}, one can simply mix and blend multiple \Verb!Osc! instances manually, which has the same effect. The code is shown in \figref{fig:wave-shaping-mk-osc}, where the \Verb!amp! parameter is expected to be in the range from zero to one, and it scans through the different spectra.

\begin{figure}[b]
\begin{lstlisting}[style=scala]
def mkOsc(freq: GE, amt: GE): GE = {
  val oddBase  = 1f
  val evenBase = 0f
  val oddDamp  = 0.7f
  val evenDamp = 0.8f
  val numHarm  = 9
  val numBufs  = 5
  val tableSz  = 1024
  
  val oscs = (0 until numBufs).map { i =>
    val amps0 = Seq.tabulate(numHarm) { j =>
      val isEven = (j + 1).isEven
      val base   = if (isEven) evenBase else oddBase
      val damp   = if (isEven) evenDamp else oddDamp
      val exp    = (j / 2) * (numBufs - i)
      base * damp.pow(exp)
    }
    // first is forced to be fundamental only
    val amps = if (i == 0) Seq(1f) else amps0
    val buf = BufferGen.sine1(amps, numFrames = tableSz)
    Osc.ar(buf, freq)
  }
  
  val idx  = amt.linlin(0, 1, 0, numBufs - 1)
  val idxF = idx.floor
  val idxC = idx.ceil
  val wC   = idx % 1.0
  val wF   = 1.0 - wC
  
  val osc  = Select.ar(idxF, oscs) * wF + Select.ar(idxC, oscs) * wC
  osc
}
\end{lstlisting}
\caption{Generation of oscillator mix implementing blending of partial frequencies.}
\label{fig:wave-shaping-mk-osc}
\end{figure}

\figref{fig:blob-shaper-161102-sono} shows the sonogram of a bounce of this sonification model with the QBO blob data. The bounce can be heard at \url{https://soundcloud.com/syssonproject/blob-shaper161102}. The parameters are:
%
\begin{itemize}
\item time = 2002-01-16 12:00:00Z to 2016-02-15 12:00:00Z
\item lon = 75.00 °W; lat = 2.50 °S
\item speed = 6 months/sec, mag-max = 3, min-freq = 300 Hz, max-freq  = 800 Hz
\item spread-mod-depth = 1.5, spread-mod-offset = 0.3
\end{itemize}

\begin{figure}
\includegraphics[width=\textwidth]{figures/blob-shaper161102.png}
\caption{Sonogram of QBO sonification with blob slice height mapped to overtone spectrum}
\label{fig:blob-shaper-161102-sono}
\end{figure}
%

\todo{continue here}

% \printbibliography

\end{document}